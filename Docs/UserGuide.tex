\documentclass[11pt]{article}

\input{WebMacros}
\input{SMacros}

\begin{document}

\section{Publishing a COM Class}
At its simplest, a COM object in R is a named collection of functions.
Clients identify the objects methods using the names in the S list. In
R, we can use lexical scoping and closures to allow some or all of
these functions to share mutable state, i.e. variables that change
across calls and are accessible to the different functions.  In this
case, we typically create the list of functions for each new object in
order to get a different environment that is independent of other
objects. But again, we are dealing simply with a list of functions.

In addition to the functions or methods, a COM object can also have
properties that it makes accessible to clients.  At its simplest, the
properties need only be named.  And if we are using lexical scoping,
they are naturally stored in the environment shared by the functions.

There are a variety of other ways to define an S-COM class and arrange
instances of such definitions. However, the \textit{simplest} approach
is to use closures and specify properties by name.  This involves
defining a constructor function associated with the S-COM class.  This
is used to create the list of functions that act as methods.  We also
specify the names of the properties whose values can be found in the
environment of these functions.  Next we specify a name for the S-COM
class. And if we are being well-behaved, we specify a collection of
strings providing a short description for each of the methods.

Let's take a look at a simple example.  We will define a COM class to
describe a Normal distribution. It has two properties: its mean and
standard deviation. And it provides methods for generating a sample of
$n$ values, and computing quantiles, percentiles and density values.
These functions share the mean and standard deviation values across
calls. We can define a generator function (\SFunction{g}) that
provides these and the properties \Svariable{mu} and
\Svariable{sigma}.
\begin{verbatim}
g <- function(mu = 0, sigma = 1) 
{
  sample <- function(n) {
     rnorm(as.integer(n), mu, sigma)
  }

  percentile <- function(q, lower = TRUE) {
    pnorm(as.numeric(q), mu, sigma, lower.tail = as.logical(lower))
  }

  quantile <- function(p) {
    qnorm(as.numeric(p), mu, sigma)
  }

  density <- function(x) {
     dnorm(as.numeric(x), mu, sigma)
  }

  list(sample = sample, 
       percentile = percentile,
       quantile = quantile, 
       density = density,
       .properties = c("mu", "sigma"),
       .help = c(sample = "generate a sample of values",
                percentile = "CDF values from this distribution",
                quantile = "quantile values from this distribution",
                density = "values of the density function for this distribution"
                ))
}
\end{verbatim}
Now, whenever we need a new instance of this Normal distribution, we
can invoke this generator function.  Note that we return the functions
in a named list.  These functions are simple wrappers to the regular
normal distribution functions in S, and they perform appropriate
simplifications, accessing the parameters and error checking.  
These just provide a context for these existing functions.

In addition to the functions, we also specify the names of the
properties and the help in this list to round out our definition.

At this point, we have almost enough information to use this as part
of an S-COM class.  All that remains is to make this visible to client
applications.  We need to give the class a unique identifier (a UUID)
and also a regular name and then we need to put it somewhere so that
when it is needed COM and R can create a suitable object.
The steps involve in ``publishing'' this definition
are
\begin{enumerate}
\item create an appropriate type of SCOMClass obect
  to represent the definition, e.g. \SClass{SCOMIDispatch};
\item store the definition as an S object so that it can be  used
  when R is started;
\item add an entry to the Windows registry to expose
 the class identifier (UUID) and associate it with a DLL or EXE.
\end{enumerate}

These steps are achieved in S with the following commands.  We create
the definition combining the generator function and the name
(\texttt{S.Normal}) using \SFunction{SCOMIDispatch}.  This creates an
object of class \SFunction{SCOMIDispatch} that describes the class
definition.  It also generates the UUID for the class identifier.  And
next, we use \SFunction{registerCOMClass} to store the definition in
an S list that can be found when R starts up, and also add the
necessary information to the Windows registry.
\begin{verbatim}
def <- SCOMIDispatch(g, "S.Normal")
registerCOMClass(def)
\end{verbatim}

This uses the \SPackage{SWinRegistry} package\cite{SWinRegistry} to
access the Windows registry from R.  The UUIDs are generated using the
\SPackage{Ruuid} package\cite{Ruuid}.


\section{The COM Mechanism}
When a client asks for an instance of an S-COM object e.g. in Perl
\Escape{Win32::OLE->new()}, COM queries the class identifier in the
Windows registry and loads the associated RDCOMServer.dll (or EXE).
Then it asks the DLL for a factory for the specified class ID.  In our
case, we create an instance of the basic \CClass{RCOMFactory} by first
resolving the class ID within the list of the registered S-COM
classes.  Then the COM mechanism within the client asks the factory to
create an instance of the given class (using the
\CMethod{CreateInstance} method).  This ends up with a call to
\SFunction{createCOMObject} with the S-COM class definition as the
only argument.  There are different methods for this generic function,
but they are all expected to do the same general thing: create a C++
object that represents the S-COM object and knows how to lookup names
and invoke methods and access properties.

There are a variety of ways to handle these definitions and create
associated C++ objects. Basically, there are several degrees of
freedom.  We can put the computations in C++ or in S or as some
hybrid.  For better understanding and easier experimentation, we
implement the methods for looking up names, invoking methods and
accessing properties in S functions.  Good design of C++ classes
allows us to implement this flexibly and easily move between different
implementations and the different languages.  And as efficiency
becomes more important, we can move towards C++ implementations.

In the case of a \SClass{SCOMIDispatch}, the
\SFunction{createCOMObject} performs the following steps.
\begin{itemize}
\item call the generator function in the class definition
 to create the instance of the S object. This creates the
 relevant list of functions which are used as the methods.
 The environment of the first function in the list will be used to 
 find properties. 

\item   The list of methods, etc. returned by the generator
  is passed as an argument in a call to \SFunction{COMSIDispatchObject}.
  This is another generator function that provides two methods,
  one for processing COM invocations and property access,
  and a second for mapping names to integer ids.
  These methods correspond to the IDispatch methods
  \CMethod{Invoke} and \CMethod{GetIDsOfNames}.
  
\item These two functions returned from the call to \SFunction{COMSIDispatchObject}
  are passed to a C++ level constructor for the \CClass{RCOMSObject} class.
  When COM requests are processed by this object, the methods in this object pass
  their arguments to the corresponding functions. These are responsible for 


\end{itemize}

\section{Different Stategies}
If one wants to explore different ways to handle the invocations, one
can define a new method for \SFunction{createCOMObject}.  For example,
suppose we want to export top-level functions simply by name. Rather
than fetching the functions when the object is defined, we want client
calls to find the \textit{current} definition of the function.  In
this context, we have no need for lexical scoping to maintain state
across calls.  To define the COM class, all we need is the names of
the functions and the names of the properties corresponding to
top-level variables.  So we define a new S class to represent this
type of COM definition:
\begin{verbatim}
setClass("SCOMNamedFunctionClass",
	  representation("SCOMClass",
	                 functions="character",
                         properties="character"))
\end{verbatim}

Next, we create a constructor function for this class.
\begin{verbatim}
SCOMNamedFunctionClass <-
function(functions, properties = character(0), name, help = "", def = new("SCOMNamedFunctionClass"))
{
  def <- .initSCOMClass(def, name = name, help = help) 
  def@functions = functions
  def@properties = properties

  def
}
\end{verbatim}

So if a programmer wants to, for example, expose the functions
\SFunction{rnorm}, \SFunction{hist} and \SFunction{boxplot},
she would create a definition for the COM class as 
\begin{verbatim}
 def <- SCOMNamedFunctionClass(c("rnorm", "hist", "boxplot"), name = "UnivariateNormalSampler") 
\end{verbatim}
and simply register it
\begin{verbatim}
 registerCOMClass(def)
\end{verbatim}

At this point, the COM definition is accessible to clients.  They can
create an instance of the object, but we still have to define how we
will create a C++-level object to handle this object and its methods.
We do this by providing appropriate methods for the invoke and
getIDsOfNames methods at the S level.  We define a method for
\SFunction{createCOMObject} for the \SClass{SCOMNamedFunctionClass}
and have it create a suitable C++ object that will dispatch the COM
methods correctly. To do this, we can mimic the
\SFunction{COMSIDispatchObject} function in the \SPackage{RDCOMServer}
package. We pass it the names of the functions and properties to
expose and have it manage the mapping of names to integers and back,
and looking up functions and properties.  \SFunction{createCOMObject}
can the pass these two S functions to the C routine
\Croutine{R_COMSObject} to create the C++-level class.  While this all
seems rather indirect, it is actually quite simple in practice.  If we
were using the OOP package, it would amount only to extending the
invoke method in a trivial way. It is made more complicated because we
are talking about closures, etc. in ad hoc ways.

\begin{verbatim}
COMSIDispatchObject <-
#
# Generator for handling IDispatch at the S level.
# 
function(funs, properties)
{
  .ids <- computeCOMNames(funs)

  invoke <-
   function(id, method, args, namedArgs) {
     if(length(namedArgs)) {
       names(args)[1:length(namedArgs)] <- names(.ids)[namedArgs]
     }

     if(is.na(match(names(.ids)[id], names(funs))) &  any(method[-1])) {
       el <- names(.ids)[id]
       if(!is.na(match(".properties", funs))) {
          if(is.na(match(el, funs[[".properties"]]))) 
	     stop(el, " is not a property")
       } else {
          if(!exists(el, envir = environment(funs[[1]])))
	     stop(el, " is not an implicit property")
       }
         
       if(method[2]) {
          get(el, envir = environment(funs[[1]]))
       } else {
          assign(el, args[[1]], envir = environment(funs[[1]]))
       }
     } else {
       f <- funs[[id]]
       eval(as.call(c(f, args)))
     }
   }

  getIDsOfNames <- 
    function(ids) {
      i <- match(ids, names(.ids))       
      if(any(is.na(i))) {
        stop(paste("Names not found: ", paste(ids[is.na(i)], collapse=", "), " in ", paste(names(.ids), collapse=", " )))
      }

      .ids[i]
    }

  list(invoke, getIDsOfNames)	
}
\end{verbatim}

Logging commands

Simple function list.


Destructors!








\section{}
The choice of the \SClass{SCOMIDispatch} to represent the class
definition will be typical. However, there are other choices and the
choice determines how methods and properties are processed.
Essentially, the class used to represent the definition controls how
the COM instances are generated from this class.  When a client asks
for a new S COM object, the S mechanism finds the definition and then
calls \SFunction{createCOMObject}.  This is a generic function and we
have methods for different \SClass{SCOMClass} objects.  What these
methods for \SFunction{createCOMObject} do is to create the
appropriate C++-level object that handles the invocation of methods
and property access.  And these different classes process the S-COM
class definition differently.  In the case of the
\SClass{SCOMIDispatch} class, the \SFunction{createCOMObject} calls
the generator function to create the instance of the S object.  Then
it calls another generator function which is used to manage the lookup
of names and dispatch of methods and finally 

\end{document}