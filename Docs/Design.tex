\documentclass{article}
\usepackage{fullpage}
\usepackage{times}
\usepackage{hyperref}
\def\COMMethod#1{\textsl{#1}()}

\begin{document}
\section{The Main Framework}
As with all flexible pieces of software, there are many different ways
to use the R DCOM server mechanism.  One can use the R server as an
in-process or out-of-process server, i.e. embedded within the client
or running as a separate, stand-alone application.  There are reasons
to prefer each approach in different contexts.

We focus more on the elements we use to provide COM objects to clients
from within R.  We start with a simple fixed interface that is very
similar to the existing COM interface provided by Thomas Baier.  This
allows the client to 
\begin{enumerate}
\item initialize R, 
\item evaluate S expressions sent in the form of strings, 
\item source scripts,
\item call S functions,
\item assign and remove variables,
\item query the names of existing variables,
\item attach and detach libraries,
\item create COM objects.
\end{enumerate}

While this is a convenient interface for rapidly connecting to S
services, it requires that the clients understand the computational
model in S and even the syntax. Sending S commands to the R
interpreter means that Excel users, for example, must know and manage
two systems explicitly.  Not only does sending S commands require they
know the S language syntax, it also means that they must explicitly
manage the global variables within S.  This makes the code highly
non-modular and can be very complex to debug.  Instead, we want to
provide them with a more typical object model that they will be
familiar with from regular COM objects.  Here, the call methods
provided by the object's class. These methods can query and modify
local state preseved in the object. This avoids the sharing of global
state across calls and allows for multiple instances of similar
objects, potentially shared by multiple clients.  In general, this is
a much better software model in terms of maintainability, reusability,
verifiability and simply in terms of familiarity to non-S users.


The \textit{basic} infrastructure that we use is quite simple.  (We
will introduce different extensions that make it easier and more
powerful.)  Basically, an S programmer will define a COM class using S
functions and objects.  She exports this so it is available to clients
by registering it with the R COM mechanism which in turn can enter it
in the registry to be used in the regular COM manner.  This S-COM
definition is stored in an S object and available to the basic,
general COM server engine.  Basically, an S-COM class is associated
with a name and unique class identifier (UUID).  These are entered
into the registry so that clients can refer to them in order to create
an instance of this class.  The S-COM class information is stored
within the master S-COM classes definition object using this unique
class identifier.  We have a single S-COM factory class that is used
as a \textit{meta-factory}.  We create a separate instance of this
factory class for each of the different S-COM classes.  Each factory
instance is initialized with information about the S-level details of
this class (i.e. the S functions used to implement the methods of the
class).  When the factory's \COMMethod{CreateInstance} method is
called (e.g. by \Croutine{CoCreateInstanceEx} in the client), the
dynamic (IDispatch) COM object is created.  A client can get the
identifier of a particular method using \COMMethod{GetIDsOfNames}.
The S-COM mechanism queries the names of the available S functions and
returns the identifier associated with the specified name.  When the
client invokes a method in the object, the \COMMethod{Invoke} method
of the IDispatch interface knows to lookup the relevant method using
the method identifier.  The arguments are converted from
\COMType{VARIANT}s in the COM world to S objects (using the same
mechanism (and code) as in the \SPackage{RDCOMClient} package).
Similarly, the return value from the S function is converted to a
\COMType{VARIANT} and return at the conclusion of the
\COMMethod{Invoke} method.




constructor/generator functions.

properties

Exceptions

\section{Event Handlers}


\section{Details}
We build a new front-end for R which is a DLL. When loaded, this
starts the R interpreter and then loads the \SPackage{RDCOMServer}
package.  This loads the S object describing the previously registered
S-COM classes.  The routine \Croutine{DllGetClassObject} is used by
the SCM to get a factory for creating COM objects of a particular
class. When the SCM calls this, the routine converts the class
identifier (CLSID) to a string and uses this to index the S-COM class
definitions.  This element should be an object of class
\SClass{SCOMDefinition}.  The factory object is given a reference to
this object which it uses to create objects.

The \COMMethod{CreateInstance} method of the factory uses the
\SClass{SCOMDefinition} when creating the SCOM object.  We use
different derived classes to support different ways to specify class
definitions. For example, we support the basic SCOM object provided by
a collection of functions that share a common environment using the
\SClass{SCOMEnvironmentDefinition} class.  In this case, we call the
generator function to create the list of functions that operate as
methods.  Alternatively, we can use a class from the \SPackage{OOP}
package via the \SClass{SCOMOOPDefinition}.  In this case, we create
an instance of the object using the \SFunction{new} method for that
OOP class.  Similarly, if we have a read-only object, then we can use
regular functions that expect the object as the first or last
argument.  And if the SCOM object does not need to store values across
method calls, one can store simple functions or function names
and have those invoked.

We register a new SCOM class by adding it to the registry.  This is
done once when the class is registered.  This involves two steps: i)
adding the canonical COM entries to the registry, and ii) adding the
\SClass{SCOMClass} to the S object that is used to store the
collection of available COM classes.

Registering the COM identifiers with the registry is quite simple.  We
first generate a CLSID for the COM class.  This is created as a UUID,
either with \executable{guidgen} or via the \SPackage{uuid} package.
We convert this to a string.  Then we create a new key (folder) in the
registry named \verb+HKEY_CLASSES_ROOT\CLSID\+\textit{actual class
id}.  We put the convenient, client name for this class as
the default value for that 

\begin{verbatim}
createRegistryKey("CLSID", clsid)
setRegistryValue("CLSID", value = "friendly name")
setRegistryValue(paste("CLSID", "InProcServer32", sep = "\\"), value="<wherever>\\RDCOMServer.dll")
setRegistryValue(paste("CLSID", "ProgID", sep = "\\"), value=progId)

createRegistryKey(friendlyName)
setRegistryValue(paste(friendlyName, "CLSID", sep = "\\"), value=clsid)
setRegistryValue(paste(friendlyName, "CurVer", sep = "\\"), value=version)
\end{verbatim}
These steps are provided as a function and use the SWinRegistry package.
This removes the need to use the \executable{regsvr32.exe} program
to register the classes.

At this point, a client can identify an SCOM class by name (or CLSID)
and it will be found in the registry and the RDCOMServer.dll will be
loaded into the client. (Out-of-process servers not yet handled!)  The
routine \Croutine{DllGetClassObject} in the RDCOMServer.dll is invoked
when the SCM is looking for a factory for a particular COM class.  We
generate an instance of the basic factory and initialize it with the
relevant \SClass{SCOMDefinition} for that class.  Essentially, the COM
registry points the SCM to this DLL to ask for a specific class
identifier. The DLL responds only to those that are requested rather
than having to push all of the class definitions when it is loaded.


\section{S-level Control}
It is convenient to be able to customize the behavior of the COM
mechanism, both creating objects and how the lookup and dispatching of
methods is performed.  In order to allow this, we provide the generic
function \SFunction{createCOMObject} that is called by the
\Croutine{CreateInstance} of the \CClass{RCOMFunctionFactory} class.
This function is called with the S-level COM class definition
(specified when registering the class)\footnote{Do we also want the
UUID for the class?}



\section{Efficiency and IDispatch}
Clients must use the IDispatch interface.  This means that they must
ask for the id of a method before invoking it. Also, data must be
marshalled across as \COMType{VARIANT} objects.  These extra steps can
degrade performance, especially if the user-level computations are
short and the client and server are in different processes and worse
still on different machines.  If we were to compile information into
type libraries, the clients could avoid querying this information



\end{document}


